\documentclass[11pt,a4paper]{article}

% --- Packages for formatting and layout ---
\usepackage[utf8]{inputenc}
\usepackage[T1]{fontenc}
\usepackage[italian]{babel}
\usepackage{geometry}
\usepackage{parskip}
\usepackage{microtype}
\usepackage{fancyhdr}
\usepackage{lmodern}
\usepackage{mathpazo} % Better serif font for formal docs
\usepackage{graphicx}
\usepackage{hyperref}
\usepackage{titlesec}

% --- Page geometry ---
\geometry{
    top=1.5cm,
    bottom=2.5cm,
    left=2.5cm,
    right=2.5cm
}

% --- Header and footer ---


% --- Title ---
\title{\bfseries Verifiche di compliance in ambienti Cloud}
\author{Niccolò Volontè - 20642A\\
\normalsize Università degli Studi di Milano\\
\normalsize Corso di Laurea in Sicurezza dei Sistemi e delle Reti Informatiche\\
\normalsize Anno Accademico 2024/2025}
\date{9 luglio 2025}

\begin{document}

\maketitle
\vspace{1em}
\hrule
\vspace{1em}
Il presente elaborato affronta il tema della sicurezza in ambienti cloud, con particolare riferimento alle verifiche di compliance all'interno della piattaforma Amazon Web Services (AWS). 

Scopo della tesi è la progettazione e realizzazione di una suite di 13 sonde di \emph{security assurance}, capaci di eseguire controlli automatizzabili sulla configurazione delle risorse AWS, verificandone la conformità rispetto a benchmark di sicurezza, come il \emph{CIS AWS Foundations Benchmark}, il \emph{CIS Amazon EKS Benchmark} e le raccomandazioni \emph{NIST SP 800-53}.

Il lavoro si articola in quattro fasi principali:

\begin{enumerate}
  \item \textbf{Analisi dello stato dell'arte}, con studio degli standard di sicurezza (CIS, NIST, AWS Security Hub) e delle caratteristiche dei principali servizi offerti da AWS. Inoltre, comprensione del framework MoonCloud e delle sue funzionalità, che fornisce un ambiente di esecuzione e deployment per le sonde di assurance, oltre che alla gestione dei dati in input per la configurazione delle stesse.
  
  \item \textbf{Progettazione delle sonde di assurance}, traducendo i controlli descrittivi dei benchmark in logiche eseguibili. Ogni sonda è stata progettata per un servizio AWS specifico (IAM, EC2, S3, RDS, EKS, ecc.).
  
  \item \textbf{Implementazione e testing}, attraverso lo sviluppo di 13 sonde in Python usando la libreria \texttt{Boto3}, containerizzate con Docker e dotate di pipeline CI/CD per l'integrazione nel framework MoonCloud. Ogni sonda segue un'architettura standard e produce output strutturato, adatto ad una valutazione intuitiva. In totale sono stati implementati 51 controlli tra i vari servizi AWS.
  
  \item \textbf{Integrazione nella piattaforma MoonCloud}, che consente l'esecuzione delle sonde in ambienti reali, la pianificazione periodica dei controlli e la visualizzazione dei risultati globali tramite dashboard.
\end{enumerate}

Le sonde sono classificate in base alla tipologia di controlli: una parte è conforme al benchmark CIS AWS Foundations v3, un'altra estende le verifiche a componenti come Amazon EKS, SQS e Inspector, non coperti dai benchmark principali. Di particolare rilievo è la sonda \texttt{aws\_vulnerability}, che esegue analisi dinamiche sulle vulnerabilità note (CVE), offrendo una valutazione della sicurezza delle risorse AWS ECR, EC2, e Lambda.

Tutte le componenti sono state progettate con attenzione all'affidabilità, all'integrazione e alla gestione degli errori. Il lavoro ha richiesto competenze tecniche, capacità di documentazione, organizzazione e adattamento a un contesto professionale.

\vspace{1em}

\noindent
\textbf{Sviluppi futuri} prevedono l'estensione delle sonde ad altri cloud provider, l'integrazione con nuovi benchmark, e l'estensione delle sonde per coprire regioni multiple e servizi aggiuntivi.  

\vspace{2em}

\noindent
Il progetto ha contribuito ad arricchire la piattaforma MoonCloud con nuove funzionalità operative e, allo stesso tempo, ha rappresentato un'esperienza applicativa nel campo della sicurezza cloud.

\end{document}
