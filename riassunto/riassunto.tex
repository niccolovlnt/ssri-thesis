\documentclass[11pt,a4paper]{article}

% --- Packages for formatting and layout ---
\usepackage[utf8]{inputenc}
\usepackage[T1]{fontenc}
\usepackage[italian]{babel}
\usepackage{geometry}
\usepackage{parskip}
\usepackage{microtype}
\usepackage{fancyhdr}
\usepackage{lmodern}
\usepackage{mathpazo} % Better serif font for formal docs
\usepackage{graphicx}
\usepackage{hyperref}
\usepackage{titlesec}

% --- Page geometry ---
\geometry{
    top=1.5cm,
    bottom=2.5cm,
    left=2.5cm,
    right=2.5cm
}

% --- Header and footer ---


% --- Title ---
\title{\bfseries Verifiche di compliance in ambienti Cloud}

\author{Niccolò Volontè - 20642A\\[0.5em]
\normalsize Università degli Studi di Milano\\
\normalsize Corso di Laurea in Sicurezza dei Sistemi e delle Reti Informatiche\\
%
\\
\begin{minipage}[t]{0.45\textwidth}
\raggedright
\textbf{RELATORE}\\
Prof. Marco Anisetti
\end{minipage}
\hfill
\begin{minipage}[t]{0.45\textwidth}
\raggedleft
\textbf{CORRELATORE}\\
Dott. Antongiacomo Polimeno
\end{minipage}
}

\begin{document}

\maketitle
\vspace{1em}
\hrule
\vspace{1em}
Il presente elaborato affronta il tema della sicurezza in ambienti cloud, con particolare riferimento alle verifiche di compliance all'interno della piattaforma Amazon Web Services (AWS). 

Scopo della tesi è garantire l'aderenza degli ambienti cloud a benchmark di sicurezza e best practice, come il \emph{CIS AWS Foundations Benchmark}, il \emph{CIS Amazon EKS Benchmark} e le raccomandazioni \emph{NIST SP 800-53}. Per raggiungere questo obiettivo, è stata sviluppata una suite di \textbf{13} sonde di \emph{security assurance} in grado di eseguire \textbf{51} controlli automatizzabili sulla configurazione delle risorse AWS, verificandone la conformità.

Il lavoro si articola in quattro fasi principali:

\begin{enumerate}
  \item \textbf{Analisi dello stato dell'arte}, con studio degli standard di sicurezza (CIS, NIST, AWS Security Hub) e delle caratteristiche dei principali servizi offerti da AWS. Inoltre, comprensione del framework MoonCloud e delle sue funzionalità, che fornisce un ambiente di esecuzione e deployment per le sonde di assurance, oltre che alla gestione dei dati in input per la configurazione delle stesse.
  
  \item \textbf{Progettazione delle sonde di assurance}, traducendo i controlli descrittivi dei benchmark in logiche eseguibili. Ogni sonda è stata progettata per un servizio AWS specifico:
  \begin{itemize}
    \item \texttt{aws\_sqs}: \textbf{3} controlli, relativi a crittografia a riposo, tagging e policy di accesso.
    \item \texttt{aws\_inspector}: \textbf{4} controlli, relativi all'abilitazione del servizio.
    \item \texttt{aws\_iam}: \textbf{13} controlli, relativi alla configurazione delle policy IAM, gestione delle credenziali e dei permessi.
    \item \texttt{aws\_ec2}: \textbf{7} controlli, relativi alla sicurezza della rete, logging e crittografia.
    \item \texttt{aws\_s3}: \textbf{6} controlli, relativi alla verifica degli accessi, logging e sicurezza delle operazioni sui bucket.
    \item \texttt{aws\_account}: \textbf{1} controllo, relativo alla verifica del contatto di sicurezza.
    \item \texttt{aws\_config}: \textbf{1} controllo, relativo alla registrazione delle configurazioni delle risorse.
    \item \texttt{aws\_cloudtrail}: \textbf{4} controlli, relativi alle code multi-regione, crittografia e log su bucket S3.
    \item \texttt{aws\_efs}: \textbf{1} controllo, relativo alla crittografia a riposo.
    \item \texttt{aws\_kms}: \textbf{1} controllo, relativo alla rotazione delle chiavi.
    \item \texttt{aws\_rds}: \textbf{3} controlli, relativi alla verifica degli accessi, crittografia e aggiornamenti automatici.
    \item \texttt{aws\_eks}: \textbf{7} controlli, relativi all'uso di versioni supportate, crittografia a riposo, tagging e logging.
  \end{itemize}
  
  \item \textbf{Implementazione e testing}, attraverso lo sviluppo di 13 sonde in Python containerizzate con Docker e dotate di pipeline CI/CD per l'integrazione nel framework MoonCloud. Ogni sonda segue un'architettura standard e produce output strutturato, adatto ad una valutazione intuitiva.
  
  \item \textbf{Integrazione nella piattaforma MoonCloud}, che consente l'esecuzione delle sonde in ambienti reali, la pianificazione periodica dei controlli e la visualizzazione dei risultati globali tramite dashboard.
\end{enumerate}

Le sonde sono classificate in base alla tipologia di controlli: una parte è conforme al benchmark CIS AWS Foundations v3, un'altra estende le verifiche a componenti come Amazon EKS, SQS e Inspector, non coperti dai benchmark principali. Di particolare rilievo è la sonda \texttt{aws\_vulnerability}, che esegue analisi dinamiche sulle vulnerabilità note (CVE), offrendo una valutazione della sicurezza delle risorse AWS ECR, EC2, e Lambda. 

Tutte le componenti sono state progettate con attenzione all'affidabilità, all'integrazione e alla gestione degli errori. Il lavoro ha richiesto competenze tecniche, capacità di documentazione e organizzazione del lavoro.

\vspace{1em}

\noindent
\textbf{Sviluppi futuri} prevedono l'estensione delle sonde ad altri cloud provider, l'integrazione con nuovi benchmark, e l'estensione delle sonde per coprire regioni multiple e servizi aggiuntivi.  
\vspace{1em}

\end{document}
